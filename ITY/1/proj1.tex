% Project n. 1 ITY
% Jan Jakub Kubik (xkubik32)
% VUT FIT 25.2.2017


\documentclass[a4paper, 11pt, twocolumn]{article}

% --------------------------------------------------
% Reqquired options for pages size and indentations
% --------------------------------------------------
\usepackage[ text={17cm,24cm}, top=2.5cm, left=2cm]{geometry}

% Language
\usepackage[utf8x]{inputenc}
\usepackage{csquotes}
\usepackage[czech]{babel}
\usepackage{courier}
\usepackage{textcomp}

% ------------Title page--------------
\title{Typografie a publikování\\ 1.projekt}
\author{Ján Jakub Kubík\\ xkubik32@fit.vutbr.cz}
\date{}

% ----------------
% My instructions
% ----------------
\newcommand{\myuv}[1]{\quotedblbase #1\textquotedblleft} 	% my function for czech quotation marks
\newcommand\tildo{\char`\~}


\begin{document}
\maketitle

% -------
% Part 1
% -------
\section{Hladká sazba}
Hladká sazba je sazba z jednoho stupně, druhu \\a řezu pí­sma sázená na stanovenou šířku odstavce.
Skládá se z odstavců, které obvykle začínají­ za\-ráž\-kou, ale mohou být sázeny i bez 
za\-rážky – roz\-ho\-du\-jí\-­cí­ je celková grafická úprava. Odstavce jsou ukončeny východovou řádkou.
 Věty nesmějí začínat číslicí.

Barevné zvýraznění­, podtrhávání­ slov či různé velikosti písma vybraných slov se zde také 
ne\-po\-uží\-vá. Hladká sazba je určena především pro delší­ texty, jako je napří­klad beletrie. 
Porušení­ konzistence sazby působí v textu rušivě a unavuje čtenářův zrak.

% -------
% Part 2 
% -------
\section{Smíšená sazba} 

Smíšená sazba má o něco volnější­ pravidla než hladká sazba. Nejčastěji se klasická hladká 
sazba doplňuje o další řezy pí­sma pro zvýraznění­ dů\-le\-ži\-tých pojmů. Existuje \myuv{pravidlo}:	

\begin{displayquote}Čí­m ví­ce {\textbf{druhů}, \emph{řezů}},  {\scriptsize velikostí}, 
barev pí­sma a jiných efektů použijeme, tí­m profesionálněji bude  dokument vypadat. Čtenář
tím bude vždy {\Huge nadšen!}
\end{displayquote}

\textsc{Tí­mto pravidlem se \underline{nikdy} nesmí­te ří­dit}. Příliš časté zvýrazňování textových
elementů  a změny velikosti {\tiny pí­sma} jsou {\huge známkou \textbf{amatérismu}} autora a 
působí­ \emph{\textbf{velmi} rušivě}. Dobře navržený dokument nemá obsahovat ví­ce než 4 
řezy či druhy pí­sma. \texttt{Dobře navržený dokument je decentní­, ne chaotický}.

Důležitým znakem správně vysázeného dokumentu je konzistentní použí­vání­ různých druhů zvýraznění­.
To napří­klad může znamenat, že \textbf{tuč\-ný řez} pí­sma bude vyhrazen pouze pro klíčová
slova, \emph{skloněný řez} pouze pro doposud neznámé pojmy a nebude se to míchat. Skloněný 
řez ne\-pů\-so\-bí­ tak rušivě a použí­vá se častěji. V \LaTeX u jej sází­me raději pří­kazem 
\texttt{\textbackslash emph\{text\}} než \texttt{\textbackslash textit\{text\}}.

Smíšená sazba se nejčastěji používá pro sazbu vědeckých článků a technických zpráv.
 U delší­ch dokumentů vědeckého či technického charakteru je zvykem upozornit čtenáře na význam různých typů zvýraznění­ v úvodní­ kapitole.

% -------
% Part 3
% -------
\section{České odlišnosti}

Česká sazba se oproti okolní­mu světu v některých aspektech mí­rně liší­. Jednou z odlišností 
je saz\-ba uvozovek. Uvozovky se v češtině použí­vají­ převážně pro zobrazení­ pří­mé řeči. \\V menší­ 
míře se použí­vají­ také pro zvýraznění­ přezdí­vek a ironie. V češtině se použí­vá tento 
\uv{\textbf{typ uvozovek}} namí­sto anglických "uvozovek". Lze je sázet připravenými příkazy
 nebo při použití UTF-8 kódování i přímo.

Ve smíšné sazbě se řez uvozovek ří­dí­ řezem první­ho uvozovaného slova. Pokud je uvozována 
celá věta, sází­ se koncová tečka před uvozovku, pokud se uvozuje slovo nebo část věty, sází­ se tečka za uvozovku.

Druhou odlišností je pravidlo pro sázení­ konců řádků. V české sazbě by řádek neměl končit 
osa\-mo\-ce\-nou jednopí­smennou předložkou nebo spojkou. Spojkou \uv{a} končit může při sazbě do 
25 liter. Abychom \LaTeX u zabránili v sázení­ osamocených předložek, vkládáme mezi předložku
\\a slovo \textbf{nezlomitelnou mezeru} znakem \texttildelow (vlnka, tilda). Pro automatické doplnění 
vlnek slouží­ volně šiřitelný program \emph{vlna} od pana Olšáka\footnote{Viz http://petr.olsak.net/ftp/olsak/vlna/.
}.

\end{document}