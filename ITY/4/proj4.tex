% Project n. 4 (citation) ITY
% Jan Jakub Kubik (xkubik32)
% VUT FIT 15.4.2017


% -----------------------------------------
% Required packages and document formation
% -----------------------------------------
\documentclass[a4paper, 11pt, times]{article}
\usepackage[left=2cm, top=3cm, text={17cm, 24cm}]{geometry}
\providecommand{\uv}[1]{\quotedblbase #1\textquotedblleft}
\usepackage[czech]{babel}
\usepackage[utf8]{inputenc}
\usepackage[round]{natbib}


% -----------
% Title page
% -----------
\begin{document}
\begin{titlepage}
\begin{center}
  \Huge\textsc{Vysoké učení technické v~Brně} \\
      \huge\textsc{Fakulta infrmačních technologií}\\
  \vspace{\stretch{0.382}}
  \huge Typografie a publikování \---\ 4.projekt\\
  \Huge{Citace}
  \vspace{\stretch{0.618}}
\end{center}

{\LARGE \today \hfill Ján Jakub Kubík}
\end{titlepage}


% -------
% Page 1
% -------
\section*{Typografia}
Pred 564 rokmi vynaliezol Johan Gutenberg kníhtlač. Asi vtedy netušil, co úžastného sa mu podarilo
a už vôbec netušil, že jeho remeslo prežije do ďalekej budúcnosti. S kníhtlačou je veľmi úzko
spojená typografia \citep{Javorek:2008}. Typografia sa zaoberá problematikou grafickej úpravy
dokumentov s použitím vhodných rezov písma a usporiadania jednotlivých znakov a odsekov vo vhodnej,
pre čitateľa zrozumiteľnej a esteticky akceptovateľnej forme \citep{Wikipedia:2017}.

V dnešnej dobe existuje nespočitateľné množstvo rôznych druhov písma s rôznymi parametrami.
Ľudia veľmi často a veľmi radi experimentujú a preto skúšajú zdobené písma v snahe skrášliť
text, a ten sa tak často stáva neprehľadným a nečitateľným.
A preto sú tu nezdobné písma, které se používajú desaťročia. Napríklad slávný Times New Roman, ktorý
poprvé uzrel svetlo sveta 3.10.1932 \citep{Tholenaar:2010}. % 10 riadkov
Toto tvrdenie dokladajú aj slová českého grafika Zdeňka Zieglera, ktorý hovorí,
že sa všetci po čase vracajú od nových písmen ku starým osvedčeným klasikám \citep{Krc:2012}.

I veľmi obsahovo hodnotný text môže na svojej hodnote strácať práve vďaka zlej čitateľnosti
\citep{Bringhurst:The_elements_of_typhographis_style}.
Vyvážiť dokument tak, aby obsahoval hodnotné informácie a pritom bol aj príjemný na pohľad
nie je nič jednoduché. Jednou z~hlavných zásad je: \uv{\emph{Používajte maximálne dva alebo
tri druhy písma}} \citep{samara:2008}. Dôležitosť typografie si uvedomujú taktiež na
mnohých vysokých školách, ako napr. na VUT FIT v Brně. Kde už v akademickom roku 2005/2006 bol
vytvorený predmet Typografie a publikování \citep{VUT}.

Typografia bola v milulosti veľmi úzko spojovaná s dennou tlačou, no dnes je stále viac a viac
spojovaná s webovými stránkami \citep{Krynin:2017}.
Byť dobrým webdesignerom neznamená excelovať len v grafickej oblasti ale webdesignér musí byť taktiež oboznámený
s dobrými typografickými zásadami a praktikami.
Stav typografie na českom a slovenskom intenete hovorí o presnom opaku. Ak rozoberieme
niekoľko základných dizajnových, pravopisných a typografických pravidiel a jejich použitie na webe, tak zistíme,
že po dizajnovej stránke sú weby celkom dobré, no po typografickej sú na tom podstatne horšie \citep{Kaspar:2016}.
No nájdu sa aj výnimky \citep{Cblog:2016}.
Častým problémom webových stránok je zle použitie písma a veľmi časté gramatické chyby \citep{Cecetka:2010}.
\newpage


% ------------------
% List of citations
% ------------------
\bibliographystyle{csplainnat}
\bibliography{proj4}

\end{document}
