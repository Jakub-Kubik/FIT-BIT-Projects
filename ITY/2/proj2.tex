% Project n. 2 ITY
% Jan Jakub Kubik (xkubik32)
% VUT FIT 18.3.2017

\documentclass[a4paper, 11pt, times]{article}

% ------------------------------------------
% Required pages size, options and packages
% ------------------------------------------
\usepackage[left=1.5cm, top=2.5cm, text={18cm, 25cm}]{geometry}
\usepackage[utf8]{inputenc}
\usepackage[czech]{babel}
\usepackage{times}
\usepackage{mathptmx}

\usepackage{amsmath}
\usepackage{amsthm}
\usepackage{amssymb}

% For dedinitions in 1. section -----------------------------
\theoremstyle{definition}
\newtheorem{definition}{Definice}[section]

\theoremstyle{theorem}
\newtheorem{theorem}[definition]{Algoritmus}
\newtheorem{lema}{Věta}

\theoremstyle{remark}
\newtheorem*{remark}{Důkaz}

% ----------
% Tite page 
% ----------
\begin{document}
\begin{center}
  \Huge\textsc{Fakulta infrmačních technologií \\
      Vysoké učení technické v~Brně}\\
  \vspace{\stretch{0.382}}
  \huge Typografie a~publikování \---\ 2.projekt\\
  Sazba dokumentů a~matematických výrazů
  \vspace{\stretch{0.618}}
\end{center}

{\LARGE 2017 \hfill Ján Jakub Kubík}
\thispagestyle{empty}
\newpage

% -------
% Page 1 
% -------
\clearpage
\setcounter{page}{1}

\begin{twocolumn}
  
  % Section 0
  \section*{Úvod}
   V~této úloze si vyzkoušíme sazbu titulní strany, matematických vzorců,
   prostředí a~dalších textových struktur obvyklých pro technicky
   zaměřené texty například rovnice (\ref{sec:1}) nebo definice \ref{subsec:1} na straně \pageref{first}.
   \par Na titulní straně je využito sázení nadpisu podle optického středu s~využitím
   zlatého řezu. Tento postup byl probírán na přednášce.


  % ----------
  % Section 1
  % ----------
  \section{Matematický text}
  \label{sec:1}
  \label{first}
  Nejprve se podíváme na sázení matematických symbolů a~výrazů v~plynulém textu.
   Pro množinu $V$ označuje card$(V)$ kardinalitu $V$
  Pro množinu $V$ reprezentuje $V^*$ volný monoid generovaný množinou $V$
  s~operací konkatenace.
  Prvek identity ve volném monoidu $V^*$ značíme symbolem $\varepsilon$
  Nechť $V^+=V^*-\{\varepsilon\}$ Algebraicky je tedy $V^+$ volná pologrupa
  generovaná množinou $V$ s~operací konkatenace. Konečnou neprázdnou
  množinu $V$ nazvěme $abeceda$. Pro $\omega \in V^*$ označuje $|\omega|$ délku řetězce $\omega$
  Pro $W \subseteq V$ označuje occur$(\omega, W)$ počet výskytů
  symbolů z~$W$ v~řetězci $\omega$ a~sym$(w, i)$ určuje $i$-tý symbol řetězce $\omega$;
  například sym$(abcd, 3)=c$

  Nyní zkusíme sazbu definic a~vět s~využitím balíku
  \texttt{amsthm.}


  % Definition 1.1
  \begin{definition}  
  Definice: Bezkontextová gramatika je čtveřice $G=(V, T, P, S)$, kde $V$ je totální abeceda,
  $T \subseteq V$ je abeceda terminálů, $S \in (V-T)$ je startující symbol a~$P$ je konečná množina pravidel
  tvaru $q: A \rightarrow \alpha$, kde $A \in (V-T)$, $\alpha \in V^*$ a~$q$
  je návěští tohoto pravidla. Nechť $N=V-T$ značí abecedu neterminálů.
  Pokud $q: A \rightarrow \alpha \in P$, $\gamma$, $\delta \in V^*$, $G$,
  provádí derivační krok z~$\gamma A \delta$ do $\gamma \alpha \delta$ podle pravidla $q: A \rightarrow \alpha$,
  symbolicky píšeme $\gamma A \delta \Rightarrow \gamma \alpha \delta [q: A \rightarrow \alpha] $
  nebo zjednodušeně $\gamma A \delta \Rightarrow \gamma \alpha \delta$ .
  Standardním způsobem definujeme $\rightarrow^m$, kde $m \geq 0$ . Dále definujeme
  tranzitivní uzávěr $\Rightarrow^+$ a~tranzitivně-reflexivní uzávěr $\Rightarrow^*$ .
  \end{definition}

  Algoritmus můžeme uvádět podobně jako definice textově, nebo využít pseudokódu
  vysázeného ve vhodném prostředí (například algorithm2e).


  % Algorithm 1.2
  \begin{theorem} 
  Algoritmus pro ověření bezkontextovosti gramatiky. Mějme gramatiku G = (N, T, P, S).
  \begin{enumerate}
    \item Pro každé pravidlo $p \in P$ proveď test, zda $p$ na levé straně obsahuje právě jeden symbol z~$N$ .
    \label{itm:1}
    \item Pokud všechna pravidla splňují podmínku z~kroku \ref{itm:1}, tak je gramatika $G$ bezkontextová.
 \end{enumerate}
 \end{theorem}


  % Definition 1.3
  \begin{definition}
  \textit{Jazyk} definovaný gramatikou $G$ definujeme jako $L(G)=\{w \in T^*|S \rightarrow^* w \}$ .
  \end{definition}


  % Subsection of 1st section
  \subsection{Podsekce obsahující větu}
  \label{subsec:1}


  % Definition 1.4
  \begin{definition} 
  Nechť $L$ je libovolný jazyk. $L$ je \textit{bezkontextový jazyk}, když a~jen
  když $L=L(G)$, kde (G) je libovolná bezkontextová gramatika.
  \end{definition}


  % Definition 1.5
  \begin{definition}
  Množinu $\mathcal{L}_{CF}=\{L|L$ je bezkontextový jazyk\} nazýváme \textit{třídou bezkontextových jazyků}.
  \end{definition}


  % Sentence 1
  \begin{lema}  
  \label{fig:veta}
  Nechť $\mathcal{L}_{abc}=\{a^n b^n c^n|n \geq 0\}.$ Platí, že $L_{abc} \notin  L_{CF}$
  \end{lema}


  % Proof without number
  \begin{remark}
   Důkaz: Důkaz se provede pomocí Pumping lemma pro bezkontextové jazyky, kdy ukážeme,
   že není možné, aby platilo, což bude implikovat
  pravdivost věty \ref{fig:veta}.\hfill \qedsymbol
   \end{remark}

  
  % ----------
  % Section 2 
  % ----------
  \section{Rovnice a~odkazy}
  Složitější matematické formulace sázíme mimo plynulý text. Lze umístit několik
  výrazů na jeden řádek, ale pak je třeba tyto vhodně oddělit, například příkazem \verb|\quad|.

  \begin{align*}
  {\sqrt[x^2]{y^3_0} \quad \mathbb{N}=\{0,1,2,...\} \quad x^{y^y} \neq x^{yy} \quad z_{i_j} \not \equiv z_{ij}}
  \end{align*}

  V~rovnici (\ref{eq:1}) jsou využity tři typy závorek s~různou explicitně definovanou velikostí.

  \begin{equation} \label{eq:1} % 1. rovnica
    \begin{aligned}
    \left\{\Big[(a+b)*c\Big]^d+1\right\}\quad&=\quad x \\
    \lim_{x \to \infty} \frac{\sin^2x+\cos^2x}{4} \quad &= \quad y
    \end{aligned}
  \end{equation}\\

  V~této větě vidíme, jak vypadá implicitní vysázení limity $\lim_{n \to \infty} f(n)$ v~normálním
   odstavci textu. Podobně je to i~s~dalšími symboly jako $\sum_{1}^{n}$ či $\bigcup_{A \in B}$. V~případě
    vzorce $\lim\limits_{x \to 0} \frac{\sin x}{x} = 1$ jsme si vynutili méně úspornou sazbu příkazem
  \verb|\limits|

\begin{alignat}{2}
 \int\limits_a^b f(x)dx \quad&=\quad -\ \int_b^a f(x)dx \\
 \left(\sqrt[5]{x^4}\right)' = \left(x^\frac{4}{5}\right)' \quad&=\quad \frac{4}{5}x^{-\frac{1}{5}} = \frac{4}{5\sqrt[5]{x}} \\
 \overline{\overline{A \vee B}} \quad&=\quad \overline{\overline{A} \wedge \overline{B}}
\end{alignat}

  
% ----------
% Section 3
% ----------
\section{Matice}
Pro sázení matic se velmi často používá prostředí \verb|array| a~závorky (\verb|\left, \right|).

\begin{equation*}
  \begin{pmatrix}
    a+b & b-a  \\
    \widehat{\xi + \omega} & \hat{\pi} \\
    \vec{a} & \overleftrightarrow{AC}  \\
    0 & \beta
  \end{pmatrix}
\end{equation*}

\begin{equation*}
A=
  \begin{Vmatrix}
  \ a_{1,1} & a_{1,2} & \cdots & a_{1,n} \ \\
  \ a_{2,1} & a_{2,2} & \cdots & a_{2,n} \ \\
  \ \vdots  & \vdots  & \ddots & \vdots  \ \\
  \ a_{m,1} & a_{m,2} & \cdots & a_{m,n} \
  \end{Vmatrix}
\end{equation*}

\begin{equation*}
  \begin{vmatrix}
    \ t & u\  \\
    \ v~& w \ \\
  \end{vmatrix}
  = tw - uv
\end{equation*}

Prostředí \verb|array| lze úspěšně využít i~jinde.
\begin{equation*}
\left(\!
  \begin{array}{c}
    n \\
    k~\end{array}
\!\right) =
    \begin{cases}
      \ \frac{n!}{k!(n-k)!}       & \ \text{pro } 0 \le k~\le n\\
      \  0                        & \ \text{pro } k~< 0 \ \text{nebo}\ k~> n\\
    \end{cases}
\end{equation*}


% ----------
% Section 4
% ----------
\section{Závěrem}
V~případě, že budete potřebovat vyjádřit matematickou konstrukci
nebo symbol a~nebude se Vám dařit jej nalézt \\ v~samotném \LaTeX u,
doporučuji prostudovat možnosti balíku maker \AmS-\LaTeX.
Analogická poučka platí obecně pro jakoukoli konstrukci v~\TeX u.

\end{twocolumn}
\end{document}
